\subsection*{Methodology}

\textbf{Data Collection and Preprocessing:}
For this study, data is collected from the Kaggle human brain phantom MRI dataset\cite{opfer2023automatic}, which consists of 557 3D T1-weighted MRI sequences of the brain from a single healthy male subject. The collected data is then preprocessed to standardize the images and remove any noise or artifacts that may interfere with the training process. Common preprocessing techniques, such as resampling to a consistent resolution, intensity normalization, and noise reduction, are applied.
The data is split into a 350 image test data which will feed our diffusion model for training.

\textbf{Denoising Diffusion Probabilistic Models:}

   \textbf{Model Formulation:} Denoising Diffusion Probabilistic Models aim to learn the underlying probability distribution of the preprocessed MRI dataset using a diffusion process. Given a noise-corrupted MRI image \(X\), the denoising process is represented as:
   \[ X_t = X + \sqrt{2\gamma t} \cdot \epsilon \]
   where \(X_t\) is the denoised image at time \(t\), \(\gamma\) is the diffusion constant, and \(\epsilon\) is a Gaussian noise term.

   \textbf{Training:} The model is trained on the preprocessed MRI dataset to learn the denoising process. The objective is to minimize the denoising loss, which is typically formulated as the negative log-likelihood of the model's output with respect to the ground truth images.
   \[ \mathcal{L}_{\text{denoise}} = -\log P(X | X_t) \]

\textbf{Model Training:}
\textbf{Neural Network Architecture:} A deep neural network architecture is designed to implement the Denoising Diffusion Probabilistic Model. The network consists of multiple layers with non-linear activation functions to capture complex relationships in the data.

\textbf{Optimization:} The neural network is trained using stochastic gradient descent (SGD) or its variants, with the denoising loss as the objective function. The weights of the network are updated iteratively to minimize the denoising loss using backpropagation.

\textbf{Hyperparameter Tuning:} The learning rate, diffusion constant (\(\gamma\)), and other hyperparameters are tuned through cross-validation to achieve optimal performance.

\textbf{Dataset Augmentation:}
   Generation of Synthetic MRI Scans Using the trained Denoising Diffusion Probabilistic Model, synthetic MRI scans are generated by passing random noise samples through the network. This process creates an augmented dataset that combines the original samples with the synthetic MRI scans.

\textbf{Dataset Combination:} The augmented dataset is formed by combining the original preprocessed MRI images with the synthetic MRI scans, resulting in an expanded dataset for training the DNNs.

\textbf{Deep Neural Network Training:}
   Task-specific Architectures; Depending on the medical image analysis task (e.g., segmentation, classification), specific DNN architectures such as convolutional neural networks (CNNs) are employed.
   The DNNs are trained using task-specific objective functions, such as cross-entropy loss for classification or dice loss for segmentation, to optimize their performance on the respective tasks.

\textbf{Evaluation and Validation:}
   The trained DNN models are evaluated using appropriate performance metrics, such as accuracy, sensitivity, specificity, dice coefficient, or area under the receiver operating characteristic curve (AUC-ROC).
   To ensure the generalization capability of the models, cross-validation techniques are used to evaluate their performance on multiple subsets of the data.

\textbf{Statistical Analysis:}
   The performance of DNNs trained with the augmented dataset is compared with the baseline DNNs trained only on the raw data to assess the effectiveness of the data augmentation approach.Statistical tests, such as paired t-tests, are performed to determine the statistical significance of any performance differences observed between the two groups.

By following this comprehensive methodology, researchers can effectively leverage Denoising Diffusion Probabilistic Models to create synthetic MRI datasets in Uganda. The proposed approach addresses the challenge of limited local data, leading to the development of more accurate and robust DNN models for medical image analysis. The results of this study have the potential to contribute significantly to improved healthcare practices and patient outcomes in the region.
