\begin{abstract}
    This research investigates the potential of generating a more extensive image dataset to tackle the data scarcity issue in training Deep Neural Networks (DNNs) for medical image analysis in Uganda. The proposed methodology involves conducting experiments with a small external data sample and employing Denoising Diffusion Probabilistic Models to create synthetic images. By integrating these synthetic images with the original dataset, a new DNN model is trained. The primary objective is to comprehensively assess the model's performance when trained on the augmented dataset, with a focus on evaluating its generalization capability and accuracy in producing reliable results for Uganda's distinct medical imaging data. The successful implementation of this approach could substantially alleviate the challenges posed by limited data in Uganda and potentially offer practical solutions for enhancing medical image analysis in other resource-constrained regions. However, to confirm its effectiveness, further validation and rigorous comparison with alternative methods are essential. The outcomes of this study hold significant promise for advancing healthcare practices and improving patient outcomes in underserved regions..\\

    Keywords: Denoising Diffusion Models, Deep Neural Networks, MRI scans, Image Dataset Generation, Medical Imaging, Uganda.
\end{abstract}
