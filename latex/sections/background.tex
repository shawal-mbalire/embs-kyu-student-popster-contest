Medical imaging, and specifically Magnetic Resonance Imaging (MRI), has become an indispensable tool in modern healthcare. MRI provides non-invasive, high-resolution images of internal organs and tissues, aiding medical professionals in diagnosing and monitoring a wide range of medical conditions. Due to its use of magnetic resonance rather than radition, MRI scans are non harmful to the humans that are being imaged. Further more, MRI data can provide more data than its radiology counterparts as it can give a three dimensional scan of organs in the body. In Uganda, as in many other regions, MRI technology has played a crucial role in improving patient care and treatment outcomes.

Deep Neural Networks (DNNs) have demonstrated remarkable success in various tasks, including medical image analysis\cite{mani2020model}. By leveraging their ability to automatically learn hierarchical representations from data, DNNs have shown promising results in image segmentation, object detection, disease classification, and even image generation. Their potential to assist radiologists in accurate diagnosis and decision-making has driven significant interest in their application to medical imaging tasks.

However, the effectiveness of DNNs relies heavily on the availability of large and diverse datasets for training. These networks learn from a vast amount of data to recognize patterns, features, and correlations within the images they process. In medical imaging, access to well-curated and comprehensive datasets is critical to ensure DNNs can generalize across different patients, anatomies, and medical conditions.

In Uganda, one of the challenges faced by researchers and healthcare practitioners is the scarcity of locally available MRI datasets. Unlike well-resourced medical centers in developed countries, collecting large-scale MRI datasets in Uganda is hindered by various factors, including limited funding, fewer imaging facilities, and the associated ethical and privacy concerns related to data acquisition and sharing. Consequently, the smaller datasets available locally may not fully represent the diversity of medical conditions and population characteristics found in Uganda.

The limitation of local data poses a significant obstacle to training accurate and robust DNN models for medical image analysis in the Ugandan context. Using DNNs trained on data from other regions may not be directly applicable due to differences in imaging protocols, patient demographics, and disease prevalence. Therefore, there is a critical need to address the data scarcity issue effectively.

In recent years, generative modeling techniques, such as Denoising Diffusion Models, have shown promise in synthesizing realistic and high-quality images. Denoising Diffusion Models aim to learn the probability distribution of the data and can generate new samples that resemble the original data distribution. These models have demonstrated success in various image synthesis tasks, including natural images, art, and medical images.

The potential of Denoising Diffusion Models lies in their ability to create synthetic MRI scans that capture the characteristics of the local population and medical conditions. By leveraging these models, researchers can effectively increase the size and diversity of the MRI dataset available for training DNNs in Uganda. Consequently, this can lead to more accurate, generalized, and reliable DNN models capable of enhancing medical image analysis, diagnosis, and treatment planning in the country.

In this paper, we review the innovative use of Denoising Diffusion Models for creating synthetic MRI datasets in Uganda. By exploring the existing literature, assessing the methodology, and analyzing the results, we aim to shed light on the potential and limitations of this approach. The insights gained from this review can pave the way for developing more robust and effective DNN models in Ugandan healthcare, contributing to improved medical outcomes and patient care.
