\section*{Results}

The results section anticipates the potential outcomes of implementing the proposed methodology, which involves utilizing Denoising Diffusion Probabilistic Models to create synthetic MRI datasets for training Deep Neural Networks in Uganda. While the actual results may vary based on the dataset size, model architecture, and other factors, the expected results can be envisioned as follows:

\begin{enumerate}
    \item \textbf{Augmented MRI Dataset:} The successful implementation of Denoising Diffusion Probabilistic Models would lead to the generation of a synthetic MRI dataset that is compatible with the original local data. The augmented dataset is expected to be significantly larger and more diverse than the limited local dataset, capturing a broader range of medical conditions, anatomical variations, and patient demographics present in Uganda.

    \item \textbf{Improved Generalization:} With the augmented dataset, the trained Deep Neural Networks are expected to exhibit improved generalization performance. The models should be better equipped to handle variations and complexities in medical images from the Ugandan population, even when presented with previously unseen data. Consequently, this enhanced generalization could reduce overfitting and increase the accuracy and reliability of the DNN models for medical image analysis tasks.

    \item \textbf{Enhanced Diagnostic Accuracy:} As a result of the improved generalization and increased dataset diversity, the trained DNN models are anticipated to demonstrate higher diagnostic accuracy. For instance, in tasks such as disease classification or segmentation of organs and tissues in MRI scans, the models should be better equipped to identify and distinguish different medical conditions, leading to more precise and reliable diagnostic outcomes.

    \item \textbf{Robustness to Noise and Variability:} Denoising Diffusion Probabilistic Models inherently capture the uncertainty and noise present in the data, which can be advantageous in medical imaging scenarios. The trained DNN models are likely to demonstrate enhanced robustness to noise, artifacts, and other imaging variations, making them more resilient and consistent in their predictions.

    \item \textbf{Reduced Data Dependency:} By generating synthetic MRI scans, the proposed approach reduces the dependency on data from other countries and ensures that the DNN models are tailored to the unique characteristics of the Ugandan population. This not only addresses the challenge of data compatibility but also enhances the applicability and relevance of the models to local medical practices.

    \item \textbf{Ethical and Privacy Benefits:} Creating synthetic MRI scans provides an ethical advantage by alleviating concerns related to patient data privacy and consent. Researchers can work with de-identified and synthesized data, minimizing the risk of patient information exposure while still achieving valuable outcomes for medical research and analysis.

    \item \textbf{Potential for Advancing Healthcare:} The successful implementation of the proposed methodology has the potential to advance healthcare practices in Uganda. With more accurate and reliable DNN models, medical professionals can benefit from improved diagnostic support, better treatment planning, and enhanced patient care. This, in turn, may contribute to better health outcomes and reduced medical errors in the region.
\end{enumerate}
